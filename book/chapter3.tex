\chapter{If/Else If/Else}

Any non-trivial program will use what is called a "conditional". A conditional
is a decision. A boolean truth or falsity. One way to use a conditional is with
an \verb|if| statement.

Let's take a look at an example so that we may dissect it.


TODO:

\begin{enumerate}
\item Show the part that is the conditional so that we can reuse that
      knowledge with loops.
\item Make sure it is clear you can have an if without the others.
\item Cover boolean operators.
\item Cover implicit "truthiness" when non-zero.
\item Introduce TRUE/FALSE from glib?
\end{enumerate}

\begin{code}{ifelse1.c}
#include <stdio.h>

void main ()
{
    int a;

    print ("Please enter an integer:  ");
    scanf ("%d", &a);

    if (a > 100) {
        printf ("%d is > 100\n", a);
    } else if (a < 100) {
        printf ("%d is < 100\n", a);
    } else {
        printf ("%d is == 100\n", a);
    }
}
\end{code}

\begin{Terminal}
gcc ifelse1.c
./a.out
100 is == 100
\end{Terminal}
