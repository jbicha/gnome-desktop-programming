\chapter{Dates and Times}

\section{time\_t}

If you've spent much time around technology oriented people, you've probably
heard the term "UNIX time". This is what \ident{time\_t} is. \ident{time\_t}
is a type for storing the number of seconds since Midnight, January 1, 1970.
For example, January 1st, 2000 would have been \ident{946713600} using a
\ident{time\_t}.

To get an understanding of how this works, lets take a look at the example
below. It shows how to print a formatted \ident{time\_t} in the current
locales format.

\begin{code}{dateandtime1.c}
\input{dateandtime1}
\end{code}


\subsection{UTC vs Local time}

\section{GTimeVal}

\section{GDateTime}
